\documentclass[
    11pt,
    a4paper
]{article}
\usepackage[
    a4paper,
    left = 10mm,
    right = 10mm,
    top = 5mm,
    bottom = 5mm,
    bindingoffset = 0cm,
    columnsep = 1cm
]{geometry}

\usepackage{hyperref}
\usepackage{xparse}
\usepackage{amsmath,amssymb,amsthm}
\usepackage[utf8]{inputenc}
\usepackage[T2A]{fontenc}
\usepackage{tkz-graph}
\usepackage[russian]{babel}
\usepackage{subcaption}
\usepackage{tikz}
\usepackage{pgfplots}
\usepackage{adjustbox}
\usetikzlibrary{arrows,%
                shapes,positioning}
                
%\usepackage{nopageno,comment}
\usepackage{indentfirst}
\usepackage{cmap}
\usepackage{ifthen}
\usepackage{tikz}
\usepackage{wrapfig}
\usepackage{float}
\usepackage{lipsum}
\usepackage{listings}% http://ctan.org/pkg/listings
\lstset{
  basicstyle=\ttfamily,
  mathescape
}

\newcommand*\justify{%
  \fontdimen2\font=0.4em% interword space
  \fontdimen3\font=0.2em% interword stretch
  \fontdimen4\font=0.1em% interword shrink
  \fontdimen7\font=0.1em% extra space
  \hyphenchar\font=`\-% allowing hyphenation
}

\ExplSyntaxOn
\DeclareExpandableDocumentCommand{\eval}{m}{\fp_eval:n {#1}}
\ExplSyntaxOff

\newcommand{\Sum}{\displaystyle\sum\limits}
\newcommand{\Max}{\max\limits}
\newcommand{\Min}{\min\limits}
\newcommand{\fromto}[3]{{#1}=\overline{{#2},\,{#3}}}
\newcommand{\floor}[1]{\left\lfloor{#1}\right\rfloor}
\newcommand{\ceil}[1]{\left\lceil{#1}\right\rceil}
\newcommand{\NN}{\mathbb{N}}
\newcommand{\RR}{\mathbb{R}}
\newcommand{\tild}{\widetilde}

\newcommand{\undef}{\otimes}
\newcommand{\emptys}{\varepsilon}
\newcommand{\serial}{\texttt{serial}}
\newcommand{\diff}{\texttt{diff}}
\newcommand{\sync}{\texttt{sync}}
\newcommand{\nsync}{\texttt{near\_sync}}
\newcommand{\remove}{\texttt{remove}}
\newcommand{\lcp}{\texttt{lcp}}
\newcommand{\lce}{\texttt{LCE}}
\newcommand{\lcpinf}{\texttt{lcp}_{\infty}}
\newcommand{\start}{\texttt{start}}
\newcommand{\finish}{\texttt{end}}
\newcommand{\id}{\texttt{id}}
\newcommand{\slp}{\texttt{SLP}}
\newcommand{\rlslp}{\texttt{RLSLP}}
\newcommand{\rlslpq}{\texttt{RLSLP}_q}

\newcommand{\g}{g}
\newcommand{\grl}{g_{rl}}
\newcommand{\grlq}{g_{rlq}}
\newcommand{\zno}{z_{no}}
\newcommand\growth{{1.1}}

\newcommand{\popcnt}{\texttt{popcnt}}
\newcommand{\val}{\texttt{val}}
\newcommand{\lz}{\texttt{LZ}}
\newcommand{\pr}[2]{#1 \rightarrow #2}
\newcommand{\tspan}{\texttt{span}}
\newcommand{\parent}{\texttt{parent}}
\newcommand{\bin}{\texttt{bin}}
\newcommand{\hint}{\texttt{hint}}
\newcommand{\recompression}{\texttt{recomp}}
\newcommand{\pending}{\texttt{pending}}

\renewcommand{\le}{\leqslant}
\renewcommand{\ge}{\geqslant}
\renewcommand{\hat}{\widehat}
\renewcommand{\emptyset}{\varnothing}
\renewcommand{\epsilon}{\varepsilon}
\newcommand{\ol}{\overline}

\theoremstyle{definition}
\newtheorem{theorem}{Теорема}
\newtheorem{definition}{Определение}
\newtheorem{lemma}[theorem]{Лемма}
\newtheorem{note}[theorem]{Замечание}
\newtheorem{example}[theorem]{Пример}
\newtheorem{corollary}[theorem]{Следствие}

\newcommand{\task}[3]{\par\noindent\stepcounter{task}{\bf Задача~\arabic{task}.~\printscore{#1} {#2} } {#3} \vskip 6pt}

\newcommand*{\hm}[1]{#1\nobreak\discretionary{}{\hbox{$#1$}}{}}

\begin{document}
\centerline{\Large \bf Практика №6 по курсу <<Теория алгоритмов>>}
\centerline{\Large \bf <<Иерархия классов сложности>>}
\centerline{Группы ФТ-301, ФТ-302}

\section{Иерархия временных классов сложности}

На теории была доказана теорема об иерархии временных классов сложности:
$$\texttt{DTIME}(f(n)) \subsetneq \texttt{DTIME}(f(2n + 1)^3)$$

Отсюда следует, что $\cup_{c \geq 1}^k \texttt{DTIME}(n^c) \subsetneq P$ для любого $k$.

На самом деле можно доказать, что $\texttt{DTIME}(f(n)) \subsetneq \texttt{DTIME}(g(n))$, если $g(n) = \omega(f(n) \log f(n))$, однако данное доказательство требует более хитрых приёмов ускорения машин Тьюринга.

\subsection{Упражнение. <<Правильные>> функции}

Попробуем с помощью результата теоремы об иерархии времени ответить на следующий вопрос: существует ли вычислимая <<неправильная>> функция? (то есть функция, для которой $f(n)$ нельзя вычислить на машине Тьюринга за время $O(n + f(n))$).

Для ответа на этот вопрос попробуем построить функцию $I(n): \mathcal{N} \mapsto \{0, 1\}$. Для этого зафиксируем некоторый язык $\mathcal{L}$ над бинарным алфавитом. Также занумеруем все биарные строки натуральными числами так, чтобы для любой строки $x$ длины $\ell$ её номер $\texttt{enc}(x)$ не превосходил значения $2^{\ell + 1}$~--- для этого достаточно рассмотреть числа $1x$ в двоичной системе счисления. Тогда несложно показать, что перевод бинарной строки $x$ длины $\ell$ в соответствующую унарную строку $1^{\texttt{enc}(x)}$, а также обратную трансформацию можно произвести с помощью машины Тьюринга, работающей за время $O(\ell2^\ell) = O(\texttt{enc}(x) \log \texttt{enc}(x))$. Тогда построим функцию $I$ следующим образом: $I(n) = [\texttt{dec}(n) \in \mathcal{L}]$. Если для языка $\mathcal{L}$ существует машина Тьюринга, разрешающая его за время $T(\ell)$, то функция $I(n)$ вычисляется за время $O(T(\log n) + n \log n)$. Чтобы время на вычисление принадлежности слова языку доминировало над временем конвертации числа в двоичное представление выберем $\mathcal{L}$ из разности множеств $\texttt{DTIME}(2^{(2\ell + 1)^6}) \setminus \texttt{DTIME}(2^{\ell^2})$. Заметим, что для такого языка время вычисления функции $I(n)$ ограничено снизу величиной $\Omega(n^2)$, так как если существует машина Тьюринга, вычисляющая данную функцию за время $f(n) = O(n^2)$, то исходный язык $\mathcal{L}$ можно также разрешить за время $O(n^2) = O(2^{\ell^2})$, а значит $\mathcal{L} \in \texttt{DTIME}(2^{\ell^2})$. Таким образом $I(n)$~--- вычислимая функция, которая не является <<правильной>>.

\section{Иерархия классов}

На лекциях были рассмотрены следующие соотношения между разными классами сложности:

$$\texttt{DTIME}(f(n)) \subset_1 \texttt{NTIME}(f(n)) \subset_2 \texttt{SPACE}(f(n)) \subset_3 \texttt{NSPACE}(f(n)) \subset_4 \texttt{TIME}(k^{f(n) + \log n})$$

Нетривиальным отношениями в этой цепочке являются вложения между разными классами, то есть $\texttt{NTIME}(f(n)) \subset_2 \texttt{SPACE}(f(n))$ и $\texttt{NSPACE}(f(n)) \subset_4 \texttt{TIME}(k^{f(n) + \log n})$.
\begin{itemize}
\item[$\subset_2$] Для доказательства данного включения достаточно просимулировать работу недетерминированной машины Тьюринга, использую $O(f(n))$ ячеек памяти. Так как время работы машины Тьюиринга ограничено функцией $f(n)$, то каждая ветка исполнения имеет глубину не более чем $f(n)$, а значит можно явно сохранить все решения вдоль данного пути в дереве вычислений машины Тьюринга, используя $O(f(n))$ ячеек памяти
\item[$\subset_4$] Для доказательства данного включения необходимо рассмотреть граф всевозможных конфигураций машины Тьюринга. Так как память МТ ограничена значением $f(n)$, то количество возможных конфигураций не превосходит величины $O(k^{f(n)} \cdot n) = O(k^{f(n) + \log n})$ для некоторой константы $k$, где множитель $n$ появляться из-за присутствия в конфигурации указателя не исходную ленту, которая имеет $n$ непустых ячеек. В данном графе нужно проверить достижимость одной из терминальных вершин из вершины, соответствующей начальной конфигурации МТ. Данную задачу можно решить за полиномиальное время от размера графа, то есть за время $O((k^{f(n) + \log n})^c) = O((k^c)^{f(n) + \log n}) = O(k'^{f(n) + \log n})$.
\end{itemize}

\end{document}