\documentclass[
    11pt,
    a4paper
]{article}
\usepackage[
    a4paper,
    left = 10mm,
    right = 10mm,
    top = 5mm,
    bottom = 5mm,
    bindingoffset = 0cm,
    columnsep = 1cm
]{geometry}

\usepackage{hyperref}
\usepackage{xparse}
\usepackage{amsmath,amssymb,amsthm}
\usepackage[utf8]{inputenc}
\usepackage[T2A]{fontenc}
\usepackage{tkz-graph}
\usepackage[russian]{babel}
\usepackage{subcaption}
\usepackage{tikz}
\usepackage{pgfplots}
\usepackage{adjustbox}
\usetikzlibrary{arrows,%
                shapes,positioning}
                
%\usepackage{nopageno,comment}
\usepackage{indentfirst}
\usepackage{cmap}
\usepackage{ifthen}
\usepackage{tikz}
\usepackage{wrapfig}
\usepackage{float}
\usepackage{lipsum}
\usepackage{listings}% http://ctan.org/pkg/listings
\lstset{
  basicstyle=\ttfamily,
  mathescape
}

\newcommand*\justify{%
  \fontdimen2\font=0.4em% interword space
  \fontdimen3\font=0.2em% interword stretch
  \fontdimen4\font=0.1em% interword shrink
  \fontdimen7\font=0.1em% extra space
  \hyphenchar\font=`\-% allowing hyphenation
}

\ExplSyntaxOn
\DeclareExpandableDocumentCommand{\eval}{m}{\fp_eval:n {#1}}
\ExplSyntaxOff

\newcommand{\Sum}{\displaystyle\sum\limits}
\newcommand{\Max}{\max\limits}
\newcommand{\Min}{\min\limits}
\newcommand{\fromto}[3]{{#1}=\overline{{#2},\,{#3}}}
\newcommand{\floor}[1]{\left\lfloor{#1}\right\rfloor}
\newcommand{\ceil}[1]{\left\lceil{#1}\right\rceil}
\newcommand{\NN}{\mathbb{N}}
\newcommand{\RR}{\mathbb{R}}
\newcommand{\tild}{\widetilde}

\newcommand{\undef}{\otimes}
\newcommand{\emptys}{\varepsilon}
\newcommand{\serial}{\texttt{serial}}
\newcommand{\diff}{\texttt{diff}}
\newcommand{\sync}{\texttt{sync}}
\newcommand{\nsync}{\texttt{near\_sync}}
\newcommand{\remove}{\texttt{remove}}
\newcommand{\lcp}{\texttt{lcp}}
\newcommand{\lce}{\texttt{LCE}}
\newcommand{\lcpinf}{\texttt{lcp}_{\infty}}
\newcommand{\start}{\texttt{start}}
\newcommand{\finish}{\texttt{end}}
\newcommand{\id}{\texttt{id}}
\newcommand{\slp}{\texttt{SLP}}
\newcommand{\rlslp}{\texttt{RLSLP}}
\newcommand{\rlslpq}{\texttt{RLSLP}_q}

\newcommand{\g}{g}
\newcommand{\grl}{g_{rl}}
\newcommand{\grlq}{g_{rlq}}
\newcommand{\zno}{z_{no}}
\newcommand\growth{{1.1}}

\newcommand{\popcnt}{\texttt{popcnt}}
\newcommand{\val}{\texttt{val}}
\newcommand{\lz}{\texttt{LZ}}
\newcommand{\pr}[2]{#1 \rightarrow #2}
\newcommand{\tspan}{\texttt{span}}
\newcommand{\parent}{\texttt{parent}}
\newcommand{\bin}{\texttt{bin}}
\newcommand{\hint}{\texttt{hint}}
\newcommand{\recompression}{\texttt{recomp}}
\newcommand{\pending}{\texttt{pending}}

\renewcommand{\le}{\leqslant}
\renewcommand{\ge}{\geqslant}
\renewcommand{\hat}{\widehat}
\renewcommand{\emptyset}{\varnothing}
\renewcommand{\epsilon}{\varepsilon}
\newcommand{\ol}{\overline}

\theoremstyle{definition}
\newtheorem{theorem}{Теорема}
\newtheorem{definition}{Определение}
\newtheorem{lemma}[theorem]{Лемма}
\newtheorem{note}[theorem]{Замечание}
\newtheorem{example}[theorem]{Пример}
\newtheorem{corollary}[theorem]{Следствие}

\newcommand{\task}[3]{\par\noindent\stepcounter{task}{\bf Задача~\arabic{task}.~\printscore{#1} {#2} } {#3} \vskip 6pt}

\newcommand*{\hm}[1]{#1\nobreak\discretionary{}{\hbox{$#1$}}{}}

\begin{document}
\centerline{\Large \bf Практика №4 по курсу <<Теория алгоритмов>>}
\centerline{\Large \bf <<Альтернативные модели вычислений>>}
\centerline{Группы ФТ-301, ФТ-302}

Разберём упражнение с одной из первых практик: является ли множество разрешимых языков замкнутым относительно операции гомоморфизма?

Напомним, что гоморфизм языка $\mathcal{L} \subset \Sigma^*$ задается функцией $\mathbf{h}: \Sigma \mapsto \Delta^*$ таким образом, что для любого слова $w \in \mathcal{L}$ значение $\mathbf{h}(w)$ определяется как конкатенация образов букв слова $w = c_1 c_2 \cdots c_n: \mathbf{h}(w) = \mathbf{h}(c_1) \cdot \mathbf{h}(c_1) \cdots \mathbf{h}(c_n)$.

Несложно показать, что множество перечислимых языков замкнуто относительно операции гомоморфизма. Действительно, пусть есть перечислимый язык $\mathcal{L}$ и дана МТ $M$, перечисляющая слова этого языка. Тогда можно построить $M'$ для перечисления слов $\mathbf{h}(\mathcal{L})$ следующим образом: запустим МТ $M$ и при появлении очередного слова $w \in \mathcal{L}$ запустим процедуру применения гомоморфизма $\mathbf{h}(w)$, после чего сравним получившееся слово с уже перечисленными ранее и в случае его уникальности напечатаем $\mathbf{h}(w)$ на ленте.

Заметим также, что несложно показать, что множество разрешимых языков замкнуто относительно операции \textbf{нестирающего} гомоморфизма, то есть любой разрешимый язык $\mathcal{L}$ под действием функции $\mathbf{h}: \Sigma \mapsto \Delta^+$ переходит в разрешимый язык. Для этого достаточно заметить, что длина образа нестирающего гомоморфизма всегда не меньше, чем длина исходного слова $|w| \leq |\mathbf{h}(w)|$, а это значит, что для проверки принадлежности слова $w'$ языку $\mathbf{h}(\mathcal{L})$ достаточно перебрать все слова $w \in \mathcal{L}$ не длиннее чем $|w'|$ и проверить, могли ли они являться прообразами для слова $w'$.

Для доказательства того, что множество разрешимых языков не замкнуто относительно операции \textbf{стирающего} гомоморфизма, можно построить следующий контрпример. Рассмотрим разрешимый язык $\mathcal{L} = \{ \langle M, x, t \rangle \mid \text{машина Тьюринга } M \text{ останавливается на входе } x \text{ за } t \text{ шагов } \}$. Пусть алфавит данного языка состоит из трёх непересекающихся множеств символов $\Sigma = \Sigma_{M} \cup \Sigma_{x} \cup \Sigma_{t}$, где каждая часть используется для кодирования конкретной компоненты кортежа. Тогда если рассмотреть гомоморфизм $\mathbf{h}(x) = \begin{cases}\varepsilon, & x \in \Sigma_{t} \\ x, & \text{иначе}\end{cases}$, то получим язык $\mathbf{h}(\mathcal{L}) = \{ \langle M, x \rangle \mid \text{машина Тьюринга } M \text{ останавливается на входе } x \}$, который не является разрешимым.

\section{Альтернативные модели вычислений}

Помимо частично рекурсивных функций и машин Тьюринга есть другие модели, созданные примерно в то же время для формализации понятия алгоритм. Рассмотрим ещё пару формализмов таких как алгорифмы и грамматики.

\subsection{Алгорифмы (нормальные алгоритмы Маркова)}

Нормальный алгоритм Маркова задается упорядоченным списком $R$, состоящим из обычных правил $\mathcal{L} \rightarrow \mathcal{D}$ или же терминальных правил $\mathcal{L} \rightarrow \cdot \mathcal{D}$. Алгорифм с заданными списком правил работает итеративно, где применение одной итерации к слову $w$ осуществляется следующим образом:
\begin{itemize}
\item Если среди правил нет такого, что левая часть $\mathcal{L}$ входит в $w$ как подслово, то алгорифм заканчивает свою работу и его выходом является слово $w$
\item Иначе, находится первое правило из упорядоченного списка $R$ такое, что его левая часть $\mathcal{L}$ входит в $w$ как подслово
\item Выбирается самое левое вхождение левой части в $w$, то есть такое разбиение $w = A \cdot \mathcal{L} \cdot B$, что $|A|$~--- минимален
\item Результатом данной итерации является слово $w' = A \cdot \mathcal{D} \cdot B$
\item Если выбранное правило было терминальным, то алгорифм заканчивает свою работу и его выходом является слово $w'$
\end{itemize}

\subsubsection{Пример. Удаление повторящихся букв}

Рассмотрим следующий алгоритм, удаляющий повторяющиеся буквы для слов над бинарным алфавитом $\{a, b\}$:
$$R = \begin{cases}aa \rightarrow a\\bb \rightarrow b\end{cases}$$

Последовательное применение алгорифма к слову $aabbba$ породит следующую цепочку промежуточных результатов:
$$aabbba \rightarrow abbba \rightarrow abba \rightarrow aba$$
где $aba$ является результатом работы алгорифма

\subsubsection{Пример. Перевод числа из бинарной системы в унарную}

Рассмотрим более содержательный пример перевода числа из бинарной системы в унарную. Поставим задачу следующим образом: дано двоичное число $w \in \{0, 1\}^+$ по которому необходимо построить слово $w' \in \{x\}^+$ такое, что $|w'| = \texttt{int(w, base: 2)}$. Например, для $w = 101$ необходимо построить $w' = xxxxx$.

Одна из возможных структур алгорифма может выглядеть следующим образом: 
\begin{itemize}
\item Заменим все $1$ на подстроки $0x$
\item Каждую пару $x0$ можно преобразовать в подстроку $0xx$, соответствующую переносу слагаемого $2^i$ в предыдущий разряд $i-1$
\item Удалить все нули
\end{itemize}

Если записывать данное описание в виде правил алгорифма получится следующий список:
$$R_1 = \begin{cases}1 \rightarrow 0x\\x0 \rightarrow 0xx\\ 0 \rightarrow \varepsilon\end{cases}$$

Ход вычислений для строки $101$ будет выглядеть следующим образом:
$$101 \rightarrow \mathbf{0x}01 \rightarrow 0x0\mathbf{0x} \rightarrow 0\mathbf{0xx}0x \rightarrow 00x\mathbf{0xx}x \rightarrow 00\mathbf{0xx}xxx \rightarrow 00xxxxx \rightarrow 0xxxxx \rightarrow xxxxx$$

Несложно доказать корректность работы алгорифма:
\begin{itemize}
\item Ясно, что сначала будет применено первое правило, после чего в строке не останется единиц и первое правило больше никогда не будет применяться
\item Для каждого $x$-а из строки $w' \in \{0, x\}^+$ определим его <<вес>> как $2^z$, где $z$~--- это количество нулей после этого $x$-а. Например, в строке $000x0xx0x$ получатся следующие веса соответственно: $[4, 2, 2, 1]$
\item Ясно, что после применения первого правила суммарный вес $x$-ов в строке будет равен значению $\texttt{int(w, base: 2)}$
\item Легко показать, что применение второго правила не меняет вес строки, так как однин $x$ с весом $2^z$ заменяется на два с весом $2^{z - 1}$
\item Так как третее правило будет применятся к строке вида $0^nx^m$ (иначе было бы применимо второе правило), то удаление нулей из неё также не меняет веса строки, а значит в конце концов получится строка $x^k$, где $k = \texttt{int(w, base: 2)}$
\end{itemize}

\subsubsection{Небольшое усложнение}

Немного усложним себе задачу и постараемся перевести число в унарную систему таким образом, чтобы финальная строка содержала символы $1$, а не вспомогательные символы $x$. То есть результатом работы алгорифма на строке $101$ должна быть строка $11111$.

Заметим, что данную модификацию задачи нельзя решить простым добавлением ко множеству правил трансформации $x \rightarrow 1$:
$$R_1' = \begin{cases}1 \rightarrow 0x\\x0 \rightarrow 0xx\\ 0 \rightarrow \varepsilon\\x \rightarrow 1\end{cases}$$

так как в результате алгорифм зациклится на любой строке, содержащей хотя бы одну единицу:
$$1 \rightarrow \mathbf{0x} \rightarrow x \rightarrow 1 \rightarrow \cdots$$

Для того, чтобы решить данную задачу построим алгорифм следующим образом:
\begin{itemize}
\item Запустим алгоритм, решающий исходную формулировку задачи
\item После его работы добавим в начало строки символ $\square$
\item Произведем замену пары символов $\square x$ на $1 \square$
\item Удалим символ $\square$
\end{itemize}

Таким образом получится следующий набор правил:
$$R_2 = \begin{cases}\square x \rightarrow 1 \square\\\square \rightarrow \cdot \varepsilon\\1 \rightarrow 0x\\x0 \rightarrow 0xx\\ 0 \rightarrow \varepsilon\\\varepsilon \rightarrow \square\end{cases}$$

Такой алгорифм будет работать следующим образом:
\begin{align*}
&101 \rightarrow \mathbf{0x}01 \rightarrow 0x0\mathbf{0x} \rightarrow 0\mathbf{0xx}0x \rightarrow 00x\mathbf{0xx}x \rightarrow 00\mathbf{0xx}xxx \rightarrow 00xxxxx \rightarrow 0xxxxx \rightarrow xxxxx \rightarrow\\
&\mathbf{\square} xxxxx \rightarrow \mathbf{1\square} xxxx \rightarrow 1\mathbf{1\square} xxx \rightarrow 11\mathbf{1\square} xx \rightarrow 111\mathbf{1\square} x \rightarrow 1111\mathbf{1\square} \rightarrow 11111
\end{align*}

\subsubsection{Теорема о сочетании алгорифмов (\textit{не было на практике})}
Не самым тривиальным образом контруируется алгорифм, являющийся результатом последовательного применения пары алгорифмов $R_1$ и $R_2$, то есть такой набор правил $R_0$, что $R_0(w) = R_2(R_1(w))$. Для упрощения доказательства будем считать, что алгорифмы $R_1$ и $R_2$ содержат \textbf{только обычные правила} и оперируют символами общего алфавита $\Sigma$. Один из возможных подходов для комбинации алгорфимов выглядит следующим образом:
\begin{itemize}
\item Сконвертируем все символы алфавита $\Sigma$ в соответствующие символы алфавита $\Sigma' = \{x' \mid x \in \Sigma\}$
\item Запустим модифицированный алгорифм $R_1$ для полученной строки
\item Сконвертируем все символы алфавита $\Sigma'$ в соответствующие символы алфавита  $\Sigma'' = \{x'' \mid x \in \Sigma\}$
\item Запустим модифицированный алгорифм $R_2$ для полученной строки
\end{itemize}

Чтобы добиться такой последовательности выполнения породим в начале строки три вспомогательных символа $\square_1, \square_2, \square_3$ и добавим правила конвертации символов между алфавитами следующим. Итоговый набор правил будет выглядеть следующим образом:
$$R_0 = \begin{cases}
\square_1 x \rightarrow x' \square_1 & \text{для всех } x \in \Sigma\\
\square_1 \rightarrow \varepsilon\\
\mathcal{L'} \rightarrow \mathcal{D'} & \text{для всех } \mathcal{L} \rightarrow \mathcal{D} \in R_1\\
\square_2 x' \rightarrow x'' \square_2 & \text{для всех } x \in \Sigma\\
\square_2 \rightarrow \varepsilon\\
\mathcal{L''} \rightarrow \mathcal{D''} & \text{для всех } \mathcal{L} \rightarrow \mathcal{D} \in R_2\\
\square_3 x'' \rightarrow x \square_3 & \text{для всех } x \in \Sigma\\
\square_3 \rightarrow \cdot \varepsilon\\
\varepsilon \rightarrow \square_3 \square_2 \square_1
\end{cases}$$

\subsection{Грамматики (\textit{не было на практике})}

Вершиной иерархии грамматик Хомского являются неограниченные грамматики, которые представляют из себя набор правил $p \rightarrow q$, где $p \in (\Sigma \cup N)^*N(\Sigma \cup N)^*, q \in (\Sigma \cup N)^*$ (то есть в левой части должен быть хотя бы один нетерминал), где $\Sigma$~--- это алфавит терминальных символов, а $N$~--- алфавит нетерминальных символов. Формально, грамматика задается четвёркой $G = \langle N, \Sigma, P, S \rangle$, где $P$~--- это множество правил $p \rightarrow q$, а $S \in N$~--- это стартовый нетерминал. Для конкретной грамматики $G$ и пары слова $x$ будем говорить, что из $x$ непосредственно выводится слово $y$ ($x \Rightarrow_G y$), если $\exists u, v, p, q \in (\Sigma \cup N)^*: (x = upv) \wedge (p \rightarrow q \in P) \wedge (y = uqv)$. Определим отношение $\Rightarrow_G^*$ как транзитивное замыкание отношения $\Rightarrow_G$. Тогда грамматика $G$ задает язык всех слов из терминальных символов, которые выводятся из стратового нетерминала $S$ за конечное число шагов, то есть $\mathcal{L}(G) = \{ w \in \Sigma^* \mid S \Rightarrow_G^* w \}$.

Множество языков, задаваемых неограниченными грамматиками совпадает с перечислимыми языками.

\subsubsection{Пример. Язык всех составных чисел}

Построим грамматику, задающую следующий язык $\mathcal{L} = \{1^{c} \mid c \text{ - составное число}\}$.

Воспользуемся для построения грамматики тем, что для любого составного числа $c$ существует пара чисел $a, b > 1$ такие, что $c = a \cdot b$. Используя этот факт построим грамматику, которая сгенерирует строку вида $A^aB^b$, после чего перенесёт все нетерминалы $B$ в левую часть, где при каждом переносе через $A$ будет добавляться новая единица в строку. Таким образом правила грамматики будут выглядеть следующим образом:
$$\begin{cases}
S \rightarrow L A A \square BB R\\
\square B \rightarrow \square BB\\
A \square \rightarrow A A \square\\
\square \rightarrow \varepsilon\\
AB \rightarrow B1A\\
1B \rightarrow B1\\
LB \rightarrow L\\
AR \rightarrow R\\
1R \rightarrow R1\\
LR \rightarrow \varepsilon
\end{cases}$$



\end{document}