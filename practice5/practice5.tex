\documentclass[
    11pt,
    a4paper
]{article}
\usepackage[
    a4paper,
    left = 10mm,
    right = 10mm,
    top = 5mm,
    bottom = 5mm,
    bindingoffset = 0cm,
    columnsep = 1cm
]{geometry}

\usepackage{hyperref}
\usepackage{xparse}
\usepackage{amsmath,amssymb,amsthm}
\usepackage[utf8]{inputenc}
\usepackage[T2A]{fontenc}
\usepackage{tkz-graph}
\usepackage[russian]{babel}
\usepackage{subcaption}
\usepackage{tikz}
\usepackage{pgfplots}
\usepackage{adjustbox}
\usetikzlibrary{arrows,%
                shapes,positioning}
                
%\usepackage{nopageno,comment}
\usepackage{indentfirst}
\usepackage{cmap}
\usepackage{ifthen}
\usepackage{tikz}
\usepackage{wrapfig}
\usepackage{float}
\usepackage{lipsum}
\usepackage{listings}% http://ctan.org/pkg/listings
\lstset{
  basicstyle=\ttfamily,
  mathescape
}

\newcommand*\justify{%
  \fontdimen2\font=0.4em% interword space
  \fontdimen3\font=0.2em% interword stretch
  \fontdimen4\font=0.1em% interword shrink
  \fontdimen7\font=0.1em% extra space
  \hyphenchar\font=`\-% allowing hyphenation
}

\ExplSyntaxOn
\DeclareExpandableDocumentCommand{\eval}{m}{\fp_eval:n {#1}}
\ExplSyntaxOff

\newcommand{\Sum}{\displaystyle\sum\limits}
\newcommand{\Max}{\max\limits}
\newcommand{\Min}{\min\limits}
\newcommand{\fromto}[3]{{#1}=\overline{{#2},\,{#3}}}
\newcommand{\floor}[1]{\left\lfloor{#1}\right\rfloor}
\newcommand{\ceil}[1]{\left\lceil{#1}\right\rceil}
\newcommand{\NN}{\mathbb{N}}
\newcommand{\RR}{\mathbb{R}}
\newcommand{\tild}{\widetilde}

\newcommand{\undef}{\otimes}
\newcommand{\emptys}{\varepsilon}
\newcommand{\serial}{\texttt{serial}}
\newcommand{\diff}{\texttt{diff}}
\newcommand{\sync}{\texttt{sync}}
\newcommand{\nsync}{\texttt{near\_sync}}
\newcommand{\remove}{\texttt{remove}}
\newcommand{\lcp}{\texttt{lcp}}
\newcommand{\lce}{\texttt{LCE}}
\newcommand{\lcpinf}{\texttt{lcp}_{\infty}}
\newcommand{\start}{\texttt{start}}
\newcommand{\finish}{\texttt{end}}
\newcommand{\id}{\texttt{id}}
\newcommand{\slp}{\texttt{SLP}}
\newcommand{\rlslp}{\texttt{RLSLP}}
\newcommand{\rlslpq}{\texttt{RLSLP}_q}

\newcommand{\g}{g}
\newcommand{\grl}{g_{rl}}
\newcommand{\grlq}{g_{rlq}}
\newcommand{\zno}{z_{no}}
\newcommand\growth{{1.1}}

\newcommand{\popcnt}{\texttt{popcnt}}
\newcommand{\val}{\texttt{val}}
\newcommand{\lz}{\texttt{LZ}}
\newcommand{\pr}[2]{#1 \rightarrow #2}
\newcommand{\tspan}{\texttt{span}}
\newcommand{\parent}{\texttt{parent}}
\newcommand{\bin}{\texttt{bin}}
\newcommand{\hint}{\texttt{hint}}
\newcommand{\recompression}{\texttt{recomp}}
\newcommand{\pending}{\texttt{pending}}

\renewcommand{\le}{\leqslant}
\renewcommand{\ge}{\geqslant}
\renewcommand{\hat}{\widehat}
\renewcommand{\emptyset}{\varnothing}
\renewcommand{\epsilon}{\varepsilon}
\newcommand{\ol}{\overline}

\theoremstyle{definition}
\newtheorem{theorem}{Теорема}
\newtheorem{definition}{Определение}
\newtheorem{lemma}[theorem]{Лемма}
\newtheorem{note}[theorem]{Замечание}
\newtheorem{example}[theorem]{Пример}
\newtheorem{corollary}[theorem]{Следствие}

\newcommand{\task}[3]{\par\noindent\stepcounter{task}{\bf Задача~\arabic{task}.~\printscore{#1} {#2} } {#3} \vskip 6pt}

\newcommand*{\hm}[1]{#1\nobreak\discretionary{}{\hbox{$#1$}}{}}

\begin{document}
\centerline{\Large \bf Практика №5 по курсу <<Теория алгоритмов>>}
\centerline{\Large \bf <<Классы сложности>>}
\centerline{Группы ФТ-301, ФТ-302}

На прошлых практиках мы не задавились вопросом эффективности машин Тьюринга и смотрели на них с чисто теоретической точки как на модель, позволяющую формализовать понятие алгоритма. На этой практике задумаемся над вопросами эффективности машин Тьюринга. 

\section{Временная сложности}

Для начала определим для машины Тьюринга $M$ и входа $x$ время работы $M$ на $x$ как $T(M, x)$ равное количеству итераций работы машины Тюьринга. Будем говорить, что $M$ работает за время $T_M(n)$ на входах $x$ таких, что $|x| = n$, если $T_M(n) = \max_{x \in \Sigma^n}\{T(M, x)\}$. Так как точное значение функции $T_M(n)$ имеет сложную структуру, то имеет смысл оценивать асимптотическое поведение этой функции.

Теперь определим класс $\texttt{DTIME}(f(n))$ для некоторой функции $f: \mathcal{N} \mapsto \mathcal{R}^+$ состояющий из таких языков $\mathcal{L}$, что каждого из них существует машина Тьюринга $M$ задающая данный язык и работающая за время $T_M(n) = O(f(n))$. 

Важным классом языков является класс $\mathtt{P} = \cup_{c \geq 1} \texttt{DTIME}(n^c)$ - класс задач, решающихся за полиномиальное время от длины входа.

\section{Правильные функции}

Для доказательства теоремы об иерархии времени времени нам понадобятся специальные функции $f$ такие, что существует $k$-ленточная машина Тьюринга $M$, которая на входе вида $1^n$ выдает бинарное представление значения $f(n)$ за время $O(n + f(n))$. Мотивация рассматривать такие функции заключается в том, чтобы машина Тьюринга из класса $\texttt{DTIME}(f(n))$ могла вычислить время работы машины Тьюринга из класса $\texttt{DTIME}(g(n))$, где $g(n) \leq f(n)$. Если функция $g(n)$ не будет <<правильной>> то её вычисление может занять значительно больше времени и уже не уложится в класс $\texttt{DTIME}(f(n))$.

\subsection{Упражнение}
Ясно, что сумма <<правильных>> функций является <<правильной>> функцией. Но можно ли такое же утверждать про разность? Более формально, пусть функции $sum(n) = f_1(n) + f_2(n)$ и $f_2(n)$ являются <<правильными>>. При каких условиях можно утверждать, что $f_1(n)$ также является правильной?

Без дополнительных ограничений данное утверждение не будет верным, так как если $f_2(n) = \omega(f_1(n))$, то $f_1$ необязательно будет являться <<правильной>> функцией, не нарушая при этой <<правильность>> суммы.

Чтобы гарантировать правильность функции $f_1$ можно наложить следующее ограничение на функции: пусть существует некоторый $\varepsilon > 0$ такой, что $f_1(n) \geq \varepsilon f_2(n)$ для любого $n$. Действительно, в этом случае мы можем вычислить значение функции $f_1(n)$ путем вычисления суммы $sum(n)$, значения $f_2(n)$ и последующего линейного вычисления их разности. Весь данный процесс займет $O(f_1(n) + f_2(n) + n)$ времени, что эквивалентно $O(f_1(n) + \frac{1}{\varepsilon} f_1(n) + n) = O(f_1(n) + n)$, а значит $f_1$~--- <<правильная>> функция.

\section{Пространственная сложность}

Для того, чтобы определить пространственную сложность (количество памяти, которую используем машина Тьюринга) будем рассматривать только МТ из не менее чем трёх лент, у которых входная лента доступна только для чтения, а выходная~--- только для записи. Все остальные ленты считаются рабочими и доступны как на запись, так и на чтение. Для таких машин Тьюринга определим пространственную сложность на входе $x$ как суммарное количество ячеек всех рабочих лент, в которых машина Тьюринга записала в некоторый момент времени записала непустой символ. Обозначим пространственную сложность на конкретном входе как $S(M, x)$, а за $S_M(n)$ обозначим худший случай потребления памяти для всех входов длины $n$.

Аналогичным образом можно определить класс $DSPACE(f(n))$ как множество языков $\mathcal{L}$ таких, что для каждого существует машина Тьюринга $M$ задающая данный язык и использующая $S_M(n) = O(f(n))$ памяти.

\section{Задача проверки строки на палиндром}

Существует несколько подходов для решения задачи о проверки строки на палиндром $PAL$:
\begin{enumerate}
\item Рассмотрим МТ из двух лент, которая сначала копирует и переворачивает входную строку на вторую лента, после чего двумя указателями сравнивает противоположные символы строки. Данная машина Тьюринга решает задачу за линейное время и линейную память, а значит $PAL \in DTIME(n) \cap DSPACE(n)$
\item Альтернативный подход к решению заключается в том, чтобы в $O(\log n)$ ячейках хранить бинарное представление пары указателей на текущие символы для сравнения и постоянно перемещать каретку МТ между данными позициями, чтобы производить сравнения. Данное решение использует $O(\log n)$ памяти и работает за $O(n^2)$, а значит $PAL \in DTIME(n^2) \cap DSPACE(\log n)$
\end{enumerate}

Можно ли утверждать, что $PAL \in DTIME(n) \cap DSPACE(\log n)$? Нельзя, так как для любой МТ $M$ решающей задачу проверки строки на палиндромность верно, что $T_M(n) \cdot S_M(n) = \Omega(n^2)$ (данный момент опустили на практике, для зацепок про доказательство можно посмотреть ссылку \url{http://bit.ly/pal-time-space}).

Однако если $T_M(n) \cdot S_M(n) = \Omega(n^2)$, значит ли это что мы можем ускорить второе решение и добиться результата: $PAL \in DTIME(\frac{n^2}{\log n}) \cap DSPACE(\log n)$?

Да, можем! Общая идея решения состоит в том, что так как мы уже используем $O(\log n)$ ячеек под хранение значение указателей на символы, то можно добавить ещё $\log n$ ячеек, в котором будет хранится подотрезок длины $\log n$ исходной строки, который сейчас нужно будет сравнить с подотрезком с противоположной стороны строки. В таком случае алгоритм будет совершать $O(\frac{n}{\log n})$ сравнений блоков, каждый из которых будет выполняться за $O(n)$ (все время будет затрачено на сдвиг указатель в другую сторону строки), а значит $PAL \in DTIME(\frac{n^2}{\log n}) \cap DSPACE(\log n)$.

\end{document}