\documentclass[
    11pt,
    a4paper
]{article}
\usepackage[
    a4paper,
    left = 10mm,
    right = 10mm,
    top = 5mm,
    bottom = 5mm,
    bindingoffset = 0cm,
    columnsep = 1cm
]{geometry}

\usepackage{hyperref}
\usepackage{xparse}
\usepackage{amsmath,amssymb,amsthm}
\usepackage[utf8]{inputenc}
\usepackage[T2A]{fontenc}
\usepackage{tkz-graph}
\usepackage[russian]{babel}
\usepackage{subcaption}
\usepackage{tikz}
\usepackage{pgfplots}
\usepackage{adjustbox}
\usetikzlibrary{arrows,%
                shapes,positioning}
                
%\usepackage{nopageno,comment}
\usepackage{indentfirst}
\usepackage{cmap}
\usepackage{ifthen}
\usepackage{tikz}
\usepackage{wrapfig}
\usepackage{float}
\usepackage{lipsum}
\usepackage{listings}% http://ctan.org/pkg/listings
\lstset{
  basicstyle=\ttfamily,
  mathescape
}

\newcommand*\justify{%
  \fontdimen2\font=0.4em% interword space
  \fontdimen3\font=0.2em% interword stretch
  \fontdimen4\font=0.1em% interword shrink
  \fontdimen7\font=0.1em% extra space
  \hyphenchar\font=`\-% allowing hyphenation
}

\ExplSyntaxOn
\DeclareExpandableDocumentCommand{\eval}{m}{\fp_eval:n {#1}}
\ExplSyntaxOff

\newcommand{\Sum}{\displaystyle\sum\limits}
\newcommand{\Max}{\max\limits}
\newcommand{\Min}{\min\limits}
\newcommand{\fromto}[3]{{#1}=\overline{{#2},\,{#3}}}
\newcommand{\floor}[1]{\left\lfloor{#1}\right\rfloor}
\newcommand{\ceil}[1]{\left\lceil{#1}\right\rceil}
\newcommand{\NN}{\mathbb{N}}
\newcommand{\RR}{\mathbb{R}}
\newcommand{\tild}{\widetilde}

\newcommand{\undef}{\otimes}
\newcommand{\emptys}{\varepsilon}
\newcommand{\serial}{\texttt{serial}}
\newcommand{\diff}{\texttt{diff}}
\newcommand{\sync}{\texttt{sync}}
\newcommand{\nsync}{\texttt{near\_sync}}
\newcommand{\remove}{\texttt{remove}}
\newcommand{\lcp}{\texttt{lcp}}
\newcommand{\lce}{\texttt{LCE}}
\newcommand{\lcpinf}{\texttt{lcp}_{\infty}}
\newcommand{\start}{\texttt{start}}
\newcommand{\finish}{\texttt{end}}
\newcommand{\id}{\texttt{id}}
\newcommand{\slp}{\texttt{SLP}}
\newcommand{\rlslp}{\texttt{RLSLP}}
\newcommand{\rlslpq}{\texttt{RLSLP}_q}

\newcommand{\g}{g}
\newcommand{\grl}{g_{rl}}
\newcommand{\grlq}{g_{rlq}}
\newcommand{\zno}{z_{no}}
\newcommand\growth{{1.1}}

\newcommand{\popcnt}{\texttt{popcnt}}
\newcommand{\val}{\texttt{val}}
\newcommand{\lz}{\texttt{LZ}}
\newcommand{\pr}[2]{#1 \rightarrow #2}
\newcommand{\tspan}{\texttt{span}}
\newcommand{\parent}{\texttt{parent}}
\newcommand{\bin}{\texttt{bin}}
\newcommand{\hint}{\texttt{hint}}
\newcommand{\recompression}{\texttt{recomp}}
\newcommand{\pending}{\texttt{pending}}

\renewcommand{\le}{\leqslant}
\renewcommand{\ge}{\geqslant}
\renewcommand{\hat}{\widehat}
\renewcommand{\emptyset}{\varnothing}
\renewcommand{\epsilon}{\varepsilon}
\newcommand{\ol}{\overline}

\theoremstyle{definition}
\newtheorem{theorem}{Теорема}
\newtheorem{definition}{Определение}
\newtheorem{lemma}[theorem]{Лемма}
\newtheorem{note}[theorem]{Замечание}
\newtheorem{example}[theorem]{Пример}
\newtheorem{corollary}[theorem]{Следствие}

\newcommand{\task}[3]{\par\noindent\stepcounter{task}{\bf Задача~\arabic{task}.~\printscore{#1} {#2} } {#3} \vskip 6pt}

\newcommand*{\hm}[1]{#1\nobreak\discretionary{}{\hbox{$#1$}}{}}

\begin{document}
\centerline{\Large \bf Домашняя контрольная работа №2 по курсу <<Теория алгоритмов>>}
\centerline{\Large \bf <<Вычислительная сложность>>}
\centerline{Группы ФТ-301, ФТ-302}

Решения присылать на почту \texttt{sivukhin.nikita@yandex.ru} с темой \texttt{ТАЛГ 2020. \{Группа\}. КР 2}.

Срок сдачи решений~--- \texttt{28.12.20 16:00 ЕКБ}

\begin{enumerate}
\item Назовём машину Тьюринга $M$ <<забывчивой>>, если траектория движения её каретки не зависит от содержания входной ленты, а зависит только от длины входа (то есть для любого входа длины $n$ позиция каретки после $i$-го шага $M$ будет всегда одинаковой~--- $pos_n(i)$). Докажите, что для любой <<правильной>> функции $T: \mathbb{N} \mapsto \mathbb{N}$, если $\mathcal{L} \in \texttt{DTIME}(T(n))$, то существует <<забывчивая>> машина Тьюринга, распознающая язык $\mathcal{L}$ за время $O(T(n)^2)$.

\item Докажите, что множество языков $\mathtt{P}$ замкнуто относительно применения звезды Клини. Напомним, что $\mathcal{L}^0 = \{\varepsilon\}, \mathcal{L}^i = \{wv \mid w \in \mathcal{L}^{i-1} \wedge v \in \mathcal{L}\}$, $\mathcal{L}^* = \cup_{i \geq 0} \mathcal{L}^i$, а $\mathtt{P}^* = \{ \mathcal{L}^* \mid \mathcal{L} \in \mathtt{P}\}$.

\item Рассмотрим задачу подсчёта количества блоков из букв $a$ в строке вида $a^{i_1}\#a^{i_2}\# \cdots \# a^{i_k}, i_j > 0$ (для входа $aaa\#a\#aa\#a\#aaa$ ответом машины Тьюринга является число $5 = 101_2$, записанное в двоичной системе счисления). 
\begin{itemize}
\item Покажите, что существует машина Тьюринга с двумя лентами, которая решает данную задачу за время $O(n)$, где $n$~--- это длина входа
\item Покажите, что существует машина Тьюринга с одной лентой, которая решает данную задачу за время $O(n \log n)$
\end{itemize}

\item Пусть $\mathcal{L}_1, \mathcal{L}_2 \in \mathtt{NP} \cap \mathtt{co{\text -}NP}$. Покажите, что симметрическая разность этих языков также принадлежит пересечению классов, т.е. $\mathtt{L}_1 \oplus \mathtt{L}_2 = \{ x \mid x \text{ принадлежит ровно одному из множеств } \mathtt{L}_1, \mathtt{L}_2 \} \in \mathtt{NP} \cap \mathtt{co{\text -}NP}$.
\end{enumerate}

\end{document}