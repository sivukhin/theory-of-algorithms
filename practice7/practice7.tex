\documentclass[
    11pt,
    a4paper
]{article}
\usepackage[
    a4paper,
    left = 10mm,
    right = 10mm,
    top = 5mm,
    bottom = 5mm,
    bindingoffset = 0cm,
    columnsep = 1cm
]{geometry}

\usepackage{hyperref}
\usepackage{xparse}
\usepackage{amsmath,amssymb,amsthm}
\usepackage[utf8]{inputenc}
\usepackage[T2A]{fontenc}
\usepackage{tkz-graph}
\usepackage[russian]{babel}
\usepackage{subcaption}
\usepackage{tikz}
\usepackage{pgfplots}
\usepackage{adjustbox}
\usetikzlibrary{arrows,%
                shapes,positioning}
                
%\usepackage{nopageno,comment}
\usepackage{indentfirst}
\usepackage{cmap}
\usepackage{ifthen}
\usepackage{tikz}
\usepackage{wrapfig}
\usepackage{float}
\usepackage{lipsum}
\usepackage{listings}% http://ctan.org/pkg/listings
\lstset{
  basicstyle=\ttfamily,
  mathescape
}

\newcommand*\justify{%
  \fontdimen2\font=0.4em% interword space
  \fontdimen3\font=0.2em% interword stretch
  \fontdimen4\font=0.1em% interword shrink
  \fontdimen7\font=0.1em% extra space
  \hyphenchar\font=`\-% allowing hyphenation
}

\ExplSyntaxOn
\DeclareExpandableDocumentCommand{\eval}{m}{\fp_eval:n {#1}}
\ExplSyntaxOff

\newcommand{\Sum}{\displaystyle\sum\limits}
\newcommand{\Max}{\max\limits}
\newcommand{\Min}{\min\limits}
\newcommand{\fromto}[3]{{#1}=\overline{{#2},\,{#3}}}
\newcommand{\floor}[1]{\left\lfloor{#1}\right\rfloor}
\newcommand{\ceil}[1]{\left\lceil{#1}\right\rceil}
\newcommand{\NN}{\mathbb{N}}
\newcommand{\RR}{\mathbb{R}}
\newcommand{\tild}{\widetilde}

\newcommand{\undef}{\otimes}
\newcommand{\emptys}{\varepsilon}
\newcommand{\serial}{\texttt{serial}}
\newcommand{\diff}{\texttt{diff}}
\newcommand{\sync}{\texttt{sync}}
\newcommand{\nsync}{\texttt{near\_sync}}
\newcommand{\remove}{\texttt{remove}}
\newcommand{\lcp}{\texttt{lcp}}
\newcommand{\lce}{\texttt{LCE}}
\newcommand{\lcpinf}{\texttt{lcp}_{\infty}}
\newcommand{\start}{\texttt{start}}
\newcommand{\finish}{\texttt{end}}
\newcommand{\id}{\texttt{id}}
\newcommand{\slp}{\texttt{SLP}}
\newcommand{\rlslp}{\texttt{RLSLP}}
\newcommand{\rlslpq}{\texttt{RLSLP}_q}

\newcommand{\g}{g}
\newcommand{\grl}{g_{rl}}
\newcommand{\grlq}{g_{rlq}}
\newcommand{\zno}{z_{no}}
\newcommand\growth{{1.1}}

\newcommand{\popcnt}{\texttt{popcnt}}
\newcommand{\val}{\texttt{val}}
\newcommand{\lz}{\texttt{LZ}}
\newcommand{\pr}[2]{#1 \rightarrow #2}
\newcommand{\tspan}{\texttt{span}}
\newcommand{\parent}{\texttt{parent}}
\newcommand{\bin}{\texttt{bin}}
\newcommand{\hint}{\texttt{hint}}
\newcommand{\recompression}{\texttt{recomp}}
\newcommand{\pending}{\texttt{pending}}

\renewcommand{\le}{\leqslant}
\renewcommand{\ge}{\geqslant}
\renewcommand{\hat}{\widehat}
\renewcommand{\emptyset}{\varnothing}
\renewcommand{\epsilon}{\varepsilon}
\newcommand{\ol}{\overline}

\theoremstyle{definition}
\newtheorem{theorem}{Теорема}
\newtheorem{definition}{Определение}
\newtheorem{lemma}[theorem]{Лемма}
\newtheorem{note}[theorem]{Замечание}
\newtheorem{example}[theorem]{Пример}
\newtheorem{corollary}[theorem]{Следствие}

\newcommand{\task}[3]{\par\noindent\stepcounter{task}{\bf Задача~\arabic{task}.~\printscore{#1} {#2} } {#3} \vskip 6pt}

\newcommand*{\hm}[1]{#1\nobreak\discretionary{}{\hbox{$#1$}}{}}

\begin{document}
\centerline{\Large \bf Практика №7 по курсу <<Теория алгоритмов>>}
\centerline{\Large \bf <<Комплементарные классы сложности>>}
\centerline{Группы ФТ-301, ФТ-302}

Комплементарный класс множества языков $X$ определяется как $\mathtt{co{\text -}X} = \{\bar{\mathcal{L}} \mid \mathcal{L} \in X\}$.

Рассмотрим определение класса $\mathtt{NP}$ через понятие сертификатов: $\mathcal{L} \in \mathtt{NP} \Leftrightarrow$ существует машина Тьюринга $M$ такая, что для любого $x \in \mathcal{L}$ существует \textit{сертификат} $y$ полиномиального размера от длины $x$, то есть $|y| \leq p(|x|)$ такой, что $M(x, y) = \texttt{accept}$. Если же $x \notin \mathcal{L}$, то для любого полиномиального сертификата верно, что $M(x, y) = \texttt{reject}$.

В рамках такого определения комплементарным классом $\mathtt{co{\text -}NP}$ будет являться множество языков $\mathcal{L}$ таких, что существует машина Тьюринга $M$ такая, что  для любого $x \notin \mathcal{L}$ существует сертификат $y$ полиномиальной длины такой, что $M(x, y) = \texttt{accept}$. Если же $x \in \mathcal{L}$, то для всех полиномиальных сертификатов $M(x, y) = \texttt{reject}$.

\section{Упражнение. $\mathtt{P} = \mathtt{co{\text -}P}$}

Докажем, что классы $\mathtt{P}$ и $\mathtt{co{\text -}P}$ совпадают. Для этого возмьём произвольный язык $\mathcal{L} \in \mathtt{P}$. Для данного языка существует детерминированная машина Тьюринга $M$, распознающая его за полиномиальное время. Построим машину Тьюринга $M'$, в которой просто поменяем местами состояния $q_{accept}$ и $q_{reject}$. Несложно заметить, что $M'$ распознает язык $\bar{\mathcal{L}}$, а значит $\bar{\mathcal{L}} \in \mathtt{P}$. Отсюда следует, что $\mathtt{co{\text -}P} \subset \mathtt{P}$. Аналогичным образом можно доказать обратное включение, откуда получаем, что $\mathtt{P} = \mathtt{co{\text -}P}$.

\subsection{$\mathtt{NP}$ и $\mathtt{co{\text -}NP}$}
Как соотносятся между собой классы $\mathtt{NP}$ и $\mathtt{co{\text -}NP}$? Можно ли с ними провернуть такой же трюк для доказательства равенства, а если нет - то в чём будет проблема?

Для начала вспомним структуру недетерминированной машины Тьюринга, после чего рассмотрим определение класса $\mathtt{NP}$ через НМТ:
\begin{itemize}
\item Недетерминированная машина Тьюринга также задается шестёркой $\langle \Gamma, Q, \delta, q_{start}, q_{accept}, q_{reject} \rangle$, где самое главное отличие состоит в том, что $\delta$ является многозначным отображением ($\delta \subset (\Gamma \times Q) \times (\Gamma \times Q \times \{\leftarrow, \uparrow, \rightarrow\}$)
\item В отличие от детерминированной машины Тьюринга, НМТ имеет несколько путей вычисления, которые в совокупности образуют дерево вычислений
\item НМТ принимает вход только в том случае, если существует \textbf{хотя бы один} путь вычисления, заканчивающийся в состоянии $q_{accept}$ 
\item Время работы НМТ вычисляется как длина самого глубокого пути вычисления
\end{itemize}

Тогда класс $\mathtt{NP}$ через НМТ определяется следующим образом: $\mathcal{L} \in \mathtt{NP} \Leftrightarrow $ существует \textbf{недетерминированная} машина Тьюринга, распознающая язык $\mathcal{L}$ за полиномиальное время.

Проблема при попытке адаптировать НМТ для языка $\mathcal{L}$ к распознаванию языка $\bar{\mathcal{L}}$ возникает из-за того, что простая смена состояний $q_{accept}$ и $q_{reject}$ не приводит к какому-либо предсказемому результату, потому что если среди всех веток вычислений для слова $x$ присутствовала хотя бы одна ветка, заначивающаяся состоянием $q_{reject}$, то после смены состояний новая НМТ будет всё ещё распознавать слово $x$ исходного языка. 

Вопрос о соотношении классов $\mathtt{NP}$ и $\mathtt{co{\text -}NP}$ до сих пор остается открытым.

\subsection{Упражнение}

Решим следующую задачу: докажите, что если $\mathtt{NP} \ne \mathtt{co{\text -}NP}$, то $\mathtt{NP} \ne \mathtt{P}$.

Проведём доказательство от противного~--- пусть $\mathtt{NP} \ne \mathtt{co{\text -}NP}$ и $\mathtt{NP} = \mathtt{P}$. Но мы знаем, что $\mathtt{P} = \mathtt{co{\text -}P}$, а значит $\mathtt{NP} = \mathtt{co{\text -}NP}$~--- противоречие.

\section{Задачи связанные с проверкой на простоту}

Рассмотрим пару задач, связанных с проверкой числа на простоту:
\begin{itemize}
\item $\mathtt{IsPrime}(x)$~--- является ли число $x$ простым
\item $\mathtt{IsComposite}(x)$~--- является ли число $x$ составным
\end{itemize}

Будем считать, что исходное число $x$ записано на ленте в двоичном виде, так как в противном случае задача представляется достаточно тривиальной~--- если $x$ записано в унарной системе счисления, то можно за $O(x)$ проверить число на простоту, однако если же $x$ записано на ленте в системе счисления с основанием больше 1, то длина входа в таком случае равна $O(\log x)$ и тривиальные алгоритмы будут работать за экспоненциалное время относительно длины входа $O(x) = O(2^n), \text{ где } n = \log x$.

Легко доказать, что $\mathtt{IsComposite} \in \mathtt{NP}$, а $\mathtt{IsPrime} \in \mathtt{co{\text -}NP}$, так как для первой задачи существует сертификат $x = a \cdot b, a > 1, b > 1$ разложения на множители, а язык $\mathtt{IsPrime}$ является комплементарным к языку $\mathtt{IsComposite}$.

Более сложной задачей является доказательство соотношения $\mathtt{IsPrime} \in \mathtt{NP}$. Для этого нужно придумать достаточно компактный сертификат простоты числа $x$ (см. \url{http://bit.ly/prime-cert}).

Ещё более важный результат был получен относительно недавно~--- в 2008 году было доказано, что $\mathtt{IsPrime} \in \mathtt{P}$, а значит и $\mathtt{IsComposite} \in \mathtt{P}$.

Рассмотрим теперь задачу распознавания $\mathtt{HasBigFactor}(n, d)$~--- имеет ли число $n$ в своём разложении простой множитель $p > d$? Несложно заметить, что $\mathtt{HasBigFactor} \in \mathtt{NP}$, так как существует сертификат принадлежности числа языку - достаточно предъявить разложение $n = a \cdot p$, где $p$~--- простое и $p > d$. \textbf{Важно} понимать, что либо сертифицирующий алгоритм должен проверять $p$ на простоту, либо мы должны предоставить полиномиальный сертификат простоты числа $p$ (оба варианты будут правильными с учётом результата 2008 года). Также ясно, что $\mathtt{HasBigFactor} \in \mathtt{co{\text -}NP}$, так как в качестве отрицательного сертификата достаточно предоставить полное разложение числа на простые множители (и опять же сертифицировать каждый множитель либо проверять множители на простоту в рамках алгоритма МТ). 

Таким образом $\mathtt{HasBigFactor} \in \mathtt{NP} \cap \mathtt{co{\text -}NP}$. Данный пример интересен тем, что неизвестно, принадлежит ли данная задача классу $\mathtt{P}$ (ясно, что $\mathtt{P} \subset \mathtt{NP} \cap \mathtt{co{\text -}NP}$).
\end{document}